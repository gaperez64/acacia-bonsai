\documentclass[sigconf,screen]{acmart}
\usepackage[english]{babel}
\usepackage[ruled,vlined]{algorithm2e}
\usepackage{paralist}

\setcopyright{acmcopyright}
\copyrightyear{2018}
\acmYear{2018}
\acmDOI{10.1145/1122445.1122456}

%% These commands are for a PROCEEDINGS abstract or paper.
\acmConference[Woodstock '18]{Woodstock '18: ACM Symposium on Neural
  Gaze Detection}{June 03--05, 2018}{Woodstock, NY}
\acmBooktitle{Woodstock '18: ACM Symposium on Neural Gaze Detection,
  June 03--05, 2018, Woodstock, NY}
\acmPrice{15.00}
\acmISBN{978-1-4503-XXXX-X/18/06}

\input{define.orgtex}

\begin{document}

%%
%% The "title" command has an optional parameter,
%% allowing the author to define a "short title" to be used in page headers.
\title{Acacia-Bonsai: A Modern Implementation of Antichain-Based LTL Realizability}

\author{Michaël Cadilhac}
\email{michael@cadilhac.name}
\orcid{1234-5678-9012}
\affiliation{%
  \institution{DePaul University}
  \streetaddress{Wabash Ave}
  \city{Chicago}
  \state{IL}
  \country{USA}
  \postcode{60604}
}

\author{Guillermo A. Pérez}
\affiliation{%
  \institution{Antwerp University}
  \streetaddress{Antwerp street}
  \city{Antwerp}
  \country{Belgium}}
\email{perez@uantwerp.be}

\begin{abstract}
  We describe our implementation of antichain-based algorithms used to solve the
  realizability for linear temporal logic (LTL) specifications.  These
  algorithms were introduced by Filiot et al in the 2010s and implemented in the
  tools Acacia and Acacia+ in C and Python.  We provide a complete rewriting of
  Acacia in C++ articulated around modularity and leveraging modern techniques
  for better performances.  These techniques include compile-time specialization
  of the algorithms, the use of SIMD registers to store vectors, and several
  preprocessing steps, one of them relying on efficient Binary Decision Diagram
  (BDD) libraries.
\end{abstract}

%% Keywords. The author(s) should pick words that accurately describe
%% the work being presented. Separate the keywords with commas.
\keywords{LTL synthesis, C++, SIMD, antichains}

\maketitle

\section{Introduction}

Nowadays, hardware and software systems are everywhere around us. One way to
ensure their correct functioning is to automatically synthesize them from a
formal specification.  This has two advantages over alternatives such as
testing and model checking.  Namely, the design part of the
program-development process can be completely bypassed and the synthesized
program is correct by construction.

In this work we are be interested in synthesizing \emph{reactive
systems}~\cite{hp84}. These are systems which maintain a continuous
interaction with their environment.  Examples of reactive systems include
communication, network, and multimedia protocols as well as operating systems.
For the specification, we consider \emph{linear temporal logic} (LTL, for
short)~\cite{pnueli77}. LTL allows to naturally specify time dependence
amongst events that make up the formal specification of a system. The
popularity of LTL as a formal specification language extends to, amongst
others, artificial intelligence~\cite{gv16,cm19,gnpw20}, hybrid systems and
control~\cite{bvpyb16}, software engineering~\cite{lpb15}, and
bio-informatics~\cite{abbdfhinprs17}.

The
classical doubly-exponential-time synthesis algorithm can be decomposed
into three steps:
\begin{enumerate}
  \item \emph{compile} the input LTL formula into an automaton of exponential
    size~\cite{vw84},
  \item \emph{determinize} the automaton~\cite{safra88,piterman07} incurring a
    second exponential blowup,
  \item and determine the winner of a \emph{two-player zero-sum game} played
    on the latter automaton~\cite{pr89}.
\end{enumerate}
By and far, most alternative approaches focus on avoiding the determinization
step of the algorithm while keeping the rest of the algorithm intact.  This
has motivated the development of so-called Safra-less approaches,
e.g.~\cite{kpv06,eks16,ekrs17,tushy17} to cite a few.  Worth mentioning are
the on-the-fly game construction implemented in the Strix tool~\cite{msl18}
and the \emph{antichain}-based on-the-fly (bounded) determinization described
in~\cite{fjr09} and implemented in Acacia+~\cite{bbfjr12}. Both techniques
avoid constructing the doubly-exponential deterministic automaton.  It is
worth highlighting that Acacia+ was not ranked in recent editions of the
\emph{Reactive Synthesis Competition} (SYNTCOMP,\footnote{See
\url{http://www.syntcomp.org/}} for short) since it is no longer maintained
despite remaining one of the main temporal synthesis references for new
advancements in the field (see,
e.g.\cite{ffrt17,ztlpv17,apsec20,lms20,bltv20}).

\paragraph*{Contribution}
We provide a complete rewriting of Acacia in C++ articulated around modularity
and leveraging modern techniques for better performances. These techniques
include compile-time specialization of the algorithms, the use of SIMD
registers to store vectors, and several preprocessing steps, one of them
relying on efficient Binary Decision Diagram (BDD) libraries. For now, we have
focused on the natural decision version of the synthesis problem which takes as
input an LTL formula and asks whether a controller satisfying it (no matter
the behavior of the environment) exists. Acacia-Bonsai was specifically
designed to be able to easily try different antichain data structures as well
preprocessing pipelines.

Move "disclaimer here"

\section{Preliminaries}

\emph{Disclaimer.} For the purpose of this short abstract, we focus on the very
problem at hand and do not explore how this ties to the realizability problem;
we simply mention the latter at the end of this section, for perspective.  We do
not introduce the full definitions that are required to prove correctness of the
main algorithm.  Indeed, we strive to distillate the algorithmic essence of the
Acacia approach of the realizability problem, but refer the interested reader to
the original publications for proofs.  We also deliberately abuse the notation
therein, specializing them as needed.

Throughout this abstract, we assume the existence of two alphabets, \(I\) and
\(O\); although these stand for input and output, the actual definitions of these
two terms is more complex.

An \emph{input} is a boolean combination of symbols of \(I\) and it is \emph{pure}
if it is just a conjunction in which all the symbols in \(I\) appear (that is, it
can be seen as an element of \(2^I\)).  Similarly, an \emph{IO} is a boolean
combination of symbols of \(I \cup O\), and it is \emph{pure} if it is a conjunction
in which all the symbols in \(I \cup O\) appear.  We will generally use \(i\) to denote
inputs and \(x\) for IOs.  Two IOs \(x\) and \(y\) are \emph{compatible} if
\(x \land y\) is not a contradiction.  We note that if \(x\) and \(y\) are compatible
and \(x\) is pure, then \(x \land y = x\).

A \emph{Büchi automaton} \cA is a tuple \((Q, q_0, \delta, B)\), \(Q\) is a set of
states, \(q_0\) the initial state,
\(\delta \subseteq Q \times \bbB(I \cup O) \times Q\) is the transition relation, and
\(B \subseteq Q\) is the set of Büchi states.  The actual semantics of this automaton will
not be relevant to our exposition, we simply note that these automata are
usually defined to recognize infinite sequences of symbols from \(I \cup O\). 

We will be interested in valuations of the states of \(\cA\) that indicate some
sort of progress towards reaching Büchi states---again, we do not go into details
here.  We will simply speak of \emph{vectors over \cA} for elements in
\(\bbZ^Q\), mapping states to integers.  In practice, these vectors will range
into a finite subset of \(\bbZ\), with \(-1\) as minimum value and an upper bound
provided by the problem.

For a vector \(\vec{v}\) over \(\cA\) and an IO \(x\), we define a function that takes
one step back in the automaton, decreasing components that have seen Büchi
states.  Write \(\chi_B(q)\) for the function mapping a state \(q\) to \(1\) if \(q \in B\),
and \(0\) otherwise.  We then define:
\[\bwd(\vec{v}, x) = \left\{p \mapsto \min_{\substack{(p, y, q) \in \delta\\ x \text{ comp. with } y}} v_q -
  \chi_B(q)\right\}\enspace,\]
and we generalize this to sets using \(\bwd(S, x) = \{\bwd(\vec{v}, x) \mid \vec{v}
\in S\}\).

For a set \(S\) of vectors over \cA and a (possibly nonpure) input \(i\), define:
\[\cpre_i(S) = \bigcup_{\substack{x \text{ pure IO}\\x \text{ comp.\ with} i}} \bwd(S, x)\]

It can be proved that iterating \cpre with any possible pure input stabilizes to
a fixed point, and that the downward closure of that fixed point is independent
from the order in which the inputs are selected.  We define
\(\down\cpre^*(S)\) to be that set.

We define two vectors 

We are now equipped to define the computational problem we focus on:\\[1em]
\textbf{BackwardRealizability}
\begin{compactitem}
\item \textbf{Given:} A Büchi automaton \cA and an integer \(k > 0\),
\item \textbf{Question:} Does \(\vinit \in \down\cpre^*(\{\safe_k(\cA)\})\)?
\end{compactitem}
\vspace{1em}

We note, for completeness, that this problem is equivalent to deciding the
realizability problem associated with \cA: the question has a positive answer
iff the \emph{output player} wins the Gale-Stewart game with payoff set the
complement of the language of \(\cA\).

\section{Realizability algorithm}

The problem above admits a natural algorithmic solution: start with the initial
set, pick an input \(i\), apply \(\cpre_i\) on the set, and iterate until all inputs
induce no change to the set.  We first introduce some degrees of freedom in this
approach, then present a slight twist on that solution that will serve as a
canvas for the different optimizations.

\subsection{Boolean states}

This opportunity for optimization was identified in \cite{...}, we simply
introduce it in a more general setting and succinctly present the original
exposition when we mention how this optimization can be implemented in
Section~\ref{sec:implem-bool}.  We start with an example.  Consider the
following Büchi automaton:
% \begin{automaton}
%   \node[state] (q0) {\(q_0\)};
%   \node[state, right of=q0] (q1) {\(q_1\)};
%   \node[]...
%   \path[->] (q0) edge (q1)
%             (q1) edge (r);
% \end{automaton}

Recall that we are only interested in knowing, after \cpre has stabilized,
whether the initial state can carry a nonnegative value.  In that sense, the
crucial information associated with \(q_0\) is boolean in nature: is its value
positive or \(-1\)?  Even further, this same remark can be applied to \(q_1\), since
\(q_1\) being valued \(6\) or \(9\), for instance, is not important to the valuation
of \(q_0\).

Consequently, the set of states may be partitioned into integer-valued states
and boolean-valued ones.  Naturally, detecting which states can be made boolean
comes at a cost, and this is not necessary for correctness, hence
implementations may elect not to do it.

\subsection{Actions}

For each IO \(x\), we will have to compute \(\bwd(\vec{v}, x)\) oftentimes.  This
requires to refer to the underlying Büchi automaton and checking for each
transition therein whether \(x\) is compatible with the condition.  In general, it
will be preferable to precompute, for each \(x\), what are the relevant pairs
\((p, q)\) for which \(x\) can go from \(p\) to \(q\).  We call the set of such pairs
the \emph{action} of \(x\) and denote it \(\act(x)\); in symbols:
\[\act(x) = \{(p, q) \mid (\exists (p, y, q) \in \delta)[x \text{ is compatible with }
y]\}\enspace.\]

Further, as we will be computing \(\cpre_i(S)\) for inputs \(i\), we abstract in a
similar way the information required for this computation.  We use the term
\emph{input-action} for the set of actions of IOs compatible with \(i\) and denote
it \(\iact(i)\); in symbols:
\[\iact(i) = \{\act(x) \mid x \text{ is an IO compatible with } i\}\enspace.\]

We will also use this notation later to devise a set of sufficient inputs for
the automaton, as introduced in the previous section.

Note that from an implementation point of view, we do not require that the
actions be precomputed.  Indeed, when iterating through pairs
\((p, q) \in \act(x)\), the underlying implementation can be going back to the
automaton if it so wishes.

\subsection{Sufficient inputs}

As we consider the transitions of the Büchi automaton as being labeled by a
boolean expression, it becomes more apparent that some pure IOs can be
redundant.  For instance, consider a Büchi automaton with
\(I = \{i\}, O = \{o_1, o_2\}\), but the only transitions compatible with \(i\) are
labeled \(i \land o_1\) and \(i \land \neg o_1\).  Pure IOs compatible with the first
label will be \(i \land o_1 \land o_2\) and \(i \land o_1 \land \neg o_2\), but
certainly, these two IOs have the same actions, and optimally, we would only
consider \(i \land o_1\).  However, we should not consider \(i \land o_2\), as it is
compatible with both transitions, but does not correspond to a pure IO.  We will
thus allow our main algorithm to select certain inputs and IOs:
\begin{definition}
  An IO (\resp input) is \emph{valid} if there is a pure IO (\resp input) with
  the same action (\resp input-action).  A set \(X\) of valid IOs is
  \emph{sufficient} if it represents all the possible actions of pure IOs:
  \[\{\act(x) \mid x \in K\} = \{act(x) \mid x \text{ is a pure IO}\}\enspace.\]
  A sufficient set of inputs is defined similarly with input-actions.
\end{definition}

\subsection{Algorithm}

\begin{algorithm}
\SetKwData{Antichain}{Antichain}
\SetKwFunction{Union}{Union}\SetKwFunction{FindCompress}{FindCompress}
\SetKwInOut{Input}{input}\SetKwInOut{Output}{output}

\Input{A Büchi automaton \cA, an integer \(k > 0\)}
\Output{Whether \(\vinit \in \down \cpre^*(\{\safe_k(\cA)\})\)}
\BlankLine

Split states of \cA into boolean and nonboolean\;
Build a type \Antichain with possibly partially boolean vectors\;
Compute \vsafe: each boolean component receives 0, each nonboolean \(k - 1\)\;
Let \(S = \{\vsafe\}\) of type \Antichain\;
Compute a sufficient set \(K\) of  inputs\;
Compute the input-actions of \(K\)\;
\While{true}{
  Pick an input-action \(a\) that leads to progress in \(S\)\;
  \lIf{no such action is found}{\Return \(\vinit \in S\)}
  \(S \leftarrow \cpre_a (S)\)\;
}
\caption{Main algorithm}\label{main_algo}
\end{algorithm}

The implementation of \(\cpre_a\) is straightforward; 

- Compute a representation of 

\section{The many options at every step}


\subsection{Boolean states}

\subsection{Vectors and antichains}

\subsection{Selecting relevant inputs}

\subsubsection{Terminal inputs and IOs}

Recall our discussion on sufficient inputs of Section~\label{sec:rel}.  We
introduce the notion of \emph{terminal} IO that follows the intuition that there
are no further restriction of the IO that would lead to a more specific action:
\begin{definition}
  An IO \(x\) is said to be \emph{terminal} if for every compatible IO \(y\), we
  have \(\act(x) \subseteq \act(y)\).  Further, an \emph{input} \(i\) is said to be
  \emph{terminal} if for every compatible input \(j\) we have
  \(\iact(i) \subseteq \iact(j)\).
\end{definition}

The key property of terminal inputs is that they are automatically valid, while
still being more general than pure inputs.
\begin{proposition}
  Every pure IO is terminal and every pure input is terminal.
\end{proposition}
\begin{proof}
  Consider a pure IO \(x\) and a compatible IO \(y\).  If \((p, q) \in \act(x)\), then
  there is a transition \((p, z, q) \in \delta\) such that \(x\) is compatible with \(z\),
  and thus \(x \land z = x\).  Consequently, \(x \land z \land y = x \land y \neq \bot\),
  hence \(y\) and \(z\) are compatible and \((p, q) \in \act(y)\).  This shows that
  \(\act(x) \subseteq \act(y)\) and that \(x\) is terminal.

  Consider now a pure input \(i\) and a compatible input \(j\).  Let \(\act(x) \in
  \iact(i)\).  It holds that \(x\) is compatible with \(i\), hence \(i \land x \neq \bot\).
  Since \(i\) is pure, \(i \land j = i\), thus \(i \land j \land x \neq \bot\), and \(x\) is
  also compatible with \(j\), implying that \(\act(x) \in \iact(j)\).  This shows that
  \(\iact(i) \subseteq \iact(j)\) and that \(i\) is terminal.
\end{proof}

\begin{proposition}
  Any terminal IO and input is valid.
\end{proposition}
\begin{proof}
  We prove the case for inputs, the IO case being similar.  Let \(i\) be a
  terminal input and \(j\) be a compatible pure input (at least one exists), then
  \(\iact(i) \subseteq \iact(j)\).  Since \(j\) is pure, it is also terminal, hence
  \(\iact(j) \subseteq \iact(i)\).  Hence \(\iact(i) = \iact(j)\) and \(i\) is valid.
\end{proof}

% \begin{proposition}
%   Let \(S\) be a set of vectors; then:
%   \begin{itemize}
%   \item If \(x\) is a terminal IO, then for every compatible \emph{pure} IO \(y\),
%     \(\bwd(S, x) = \bwd(S, y)\).
%   \item If \(i\) is a terminal input, then for every compatible \emph{pure} input
%     \(j\), \(\cpre_i(S) = \cpre_j(S)\).
%   \end{itemize}
% \end{proposition}
% \begin{proof}
%   If \(x\) is a terminal IO and \(y\) is a compatible pure IO, then \(y\) is also
%   terminal, showing that \(\act(x) = \act(y)\).  Since \(\bwd(\cdot, x)\) only depends
%   on \(\act(x)\), this proves the statement.  This is similar for inputs, since
%   \(\cpre_i\) only depends on \(\iact(i)\).
% \end{proof}

For completeness, we present here a simple algorithm for computing a sufficient
set of terminal IOs.  Our most basic implementation applies this algorithm a
second time to compute a sufficient set of terminal inputs.

\begin{algorithm}
\SetKwInOut{Input}{input}\SetKwInOut{Output}{output}

\Input{A Büchi automaton \cA}
\Output{A sufficient set of terminal inputs}
\BlankLine
\(P \leftarrow \{\top\}\)\;
\For{every label \(x\) in the automaton}{
  \For{every element \(y\) in \(P\)}{
    \If{\(x \land y \neq \bot\)}{
      Delete \(y\) from \(P\)\;
      Insert \(x \land y\) in \(P\)\;
      \lIf{\(\neg x \land y \neq \bot\)}{
        insert \(\neg x \land y\) in \(P\)
      }
    }
  }
}
\Return \(P\)
\caption{Computing a sufficient set of terminal IOs}
\end{algorithm}



\subsection{Precomputing actions}

\subsection{Main loop: Picking actions}

\section{Experimental results}



\end{document}

%%
%% End of file `sample-sigconf.tex'.

% LocalWords:  Antichain LTL Realizability Michaël Cadilhac michael cadilhac
% LocalWords:  DePaul Pérez perez uantwerp SIMD antichains realizability
% LocalWords:  iteratively
