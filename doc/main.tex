\documentclass[draft]{llncs}

\usepackage{amsmath}
\usepackage{amssymb}
\usepackage{xspace}
\usepackage{todonotes}

% Usual math notation fixes
\makeatletter
\let\epsilon\varepsilon
\let\phi\varphi
\let\emptyset\varnothing
\let\rho\varrho
\makeatother

% Local macros
\newcommand{\indicator}[2]{\mathbf{1}_{#1}(#2)}
\newcommand{\pow}{\mathcal{P}}
\newcommand{\st}{\mathrel{:}}
\newcommand{\ie}{\textit{i.e.}\xspace}
\newcommand{\eve}{Eve\xspace}
\newcommand{\adam}{Adam\xspace}
\newcommand{\calN}{\mathcal{N}}
\newcommand{\calM}{\mathcal{M}}
\newcommand{\calS}{\mathcal{S}}
\newcommand{\calD}{\mathcal{D}}
\newcommand{\calG}{\mathcal{G}}
\newcommand{\lang}[1]{\mathcal{L}_{#1}}
\newcommand{\out}[2]{\pi_{#1#2}}

\begin{document}

\frontmatter

\title{Reactive Synthesis Using Universal Limit-Deterministic Co-B\"uchi
Observers}
\author{Guillermo A. P\'erez\inst{1} \and Jean-Fran\c{c}ois Raskin\inst{1}
\and Salomon Sickert\inst{2}}
\institute{Universit\'e libre de Bruxelles\\
\email{gperezme@ulb.ac.be, jraskin@ulb.ac.be} \and
Technische Universit\"at M\"unchen\\
\email{sickert@in.tum.de}}

\maketitle

\begin{abstract}
	We consider the reactive synthesis problem with specifications given in
	the form of a limit-deterministic co-B\"uchi automaton. More precisely, we
	study Gale-Stewart games played between an input and an output player
	(called \adam and \eve, respectively) and with payoff set being the
	complement of the language of a given limit-deterministic B\"uchi
	automaton: if the infinite input-output sequence is accepted by the
	automaton, then \adam wins, otherwise \eve wins.
\end{abstract}

\section{Definitions}
An \emph{infinite word} $\alpha$ over an \emph{alphabet} $A$, \ie~$\alpha \in
A^\omega$, can be seen as a function $\alpha : \mathbb{N}_{> 0} \to A$. Thus, we
write $\alpha(i)$ to refer to the $i$-th letter from $\alpha$. We also write
$\alpha(i..j)$ to denote the infix
$\alpha(i)\alpha(i+1)\dots\alpha(j-1)\alpha(j)$; $\alpha(..j)$ instead of
$\alpha(1..j)$; and $\alpha(i..)$ to denote the infinite suffix
$\alpha(i)\alpha(i+1)\ldots$

\subsection{B\"uchi and parity automata}
\begin{definition}[Infinite-word automata]
    An ($\omega$-word) \emph{automaton} is a tuple $\calN =
    (Q,q_0,A,\Delta)$ where $Q$ is a finite set of states, $q_0 \in Q$ is the
    initial state, $A$ is a finite alphabet, and $\Delta \subseteq Q \times A
    \times Q$ is the transition relation. Since we are interested in infinite
    words, we will henceforth assume that for all $p \in Q$ and all $a \in A$
    there exists $q \in Q$ such that $(p,a,q) \in Q$.
\end{definition}
A run of $\calN$ on a word $\alpha \in A^\omega$ is an infinite sequence $\rho =
q_0 \alpha(1) q_1 \alpha(2) \dots \in (Q\cdot A)^\omega$ such that
$(q_i,\alpha(i+1),q_{i+1}) \in \Delta$ for all $i \in \mathbb{N}$. The automaton
is said to be \emph{deterministic} if for all $p \in Q$ and all $a \in A$ we
have that $(p,a,q_1), (p,a,q_2) \in \Delta \implies q_1 = q_2$.

Automata are paired with a condition which determines which runs are
\emph{accepting}. In this work we will consider two conditions.
\begin{itemize}
    \item The \emph{parity} condition is defined with respect to a
        \emph{priority function} $p : Q \to \mathbb{N}$. A run $\rho = q_0 a_0
        q_1 a_1 \dots$ is accepting for the parity condition if and only if the
        value $\liminf_{i \to \infty} p(q_i)$ is even.
    \item The \emph{B\"uchi} condition is defined with respect to a set of
        \emph{B\"uchi} or accepting states $B \subseteq Q$. A run $\rho = q_0
        a_0 q_1 a_1 \dots$ is accepting for this condition if and only if for
        all $i \in \mathbb{N}$ there exists $j \ge i$ such that $q_j \in B$.
        That is, the run visits accepting states infinitely often.
\end{itemize}
A word $\alpha$ is accepted by the automaton $\calN$ if it has a run on
$\alpha$ that is accepting.  We denote by $\lang{\calN}$ the \emph{language} of
the automaton $\calN$, that is, the set of words that $\calN$ accepts.

\begin{proposition}[From~\cite{safra92,piterman07}]
    For all B\"uchi automata $\calN$ there exists a deterministic parity
    automaton $\calD$ of exponential size with respect to the size of $\calN$
    and such that $\lang{\calN} = \lang{\calD}$.
\end{proposition}

\subsection{Limit-deterministic (co-)B\"uchi automata}
\begin{definition}
    A \emph{limit-deterministic} B\"uchi automaton is a B\"uchi automaton $\calN
    = (Q,q_0,A,\Delta,B)$ with its states partitioned into \emph{deterministic
    states} $Q_d \subseteq Q$ and non-deterministic states $Q \setminus Q_d$ and
    for which the following properties hold.
    \begin{itemize}
        \item All the B\"uchi states are in $Q_d$, \ie~$B \subseteq Q_d$.
        \item For all $p \in Q_d$ and all $a \in A$ if $(p,a,q) \in \Delta$ then
            $q \in Q_d$.
        \item The transition relation $\Delta$ restricted to $Q_d$ is
            deterministic.
    \end{itemize}
\end{definition}

In this work we will consider limit-deterministic B\"uchi automata $\calN$ whose
language represent sets of ``bad'' words. Thus, rather than being interested in
the set $\lang{\calN}$, we will sometimes focus on its complement. It will
therefore be convenient to view $\calN$ as a \emph{universal limit-deterministic
co-B\"uchi automaton}. That is to say, an automaton that accepts a word $\alpha$
if all of its runs on $\alpha$ are \textbf{not} accepting with respect to the
B\"uchi condition. (In other words, all runs on $\alpha$ visit accepting states
only finitely often.)

\subsection{Synthesis games and strategies}
\begin{definition}[Games]
    A \emph{(Gale-Stewart) game} on input and output alphabets $I$ and $O$,
    respectively, is a two-player perfect-information game played by \eve and
    \adam in rounds: \adam chooses an element $i_k \in I$ and \eve responds with
    an element $o_k \in O$. A \emph{play} in such a game is an infinite word
    $\langle i_1, o_1\rangle \langle i_2, o_2\rangle \dots \in (I\times O)^\omega$. 
\end{definition}
A game is paired with a \emph{payoff set} $P \subseteq (I \times
O)^\omega$ used to determine who wins a play $\pi$. If $\pi \in P$ then
\emph{$\pi$ is winning for \eve}, otherwise it is \emph{winning for \adam}.

\begin{definition}[Strategies]
    A \emph{strategy} for \adam in a game is a function $\tau : (I \times O)^*
    \to I$ which maps every (possibly empty) play prefix
    to a choice of input letter.  Similarly, a \emph{strategy for \eve}
    is a function $\sigma : (I \times O)^* I \to O$ which maps every play prefix
    and input letter to a choice of output letter.
\end{definition}
A play $\pi = \langle i_1, o_1\rangle \langle i_2, o_2\rangle \dots$
is \emph{consistent with} a strategy $\tau$
for \adam if $i_k = \tau( \langle i_1, o_1\rangle \dots
\langle i_{k-1}, o_{k-1}\rangle)$ for all $k \in
\mathbb{N}$; it is consistent with a strategy $\sigma$ for \eve if $o_k =
\sigma( \langle i_1, o_1\rangle \dots \langle i_{k-1}, o_{k-1}\rangle, i_{k})$.
A pair of strategies $\sigma$ and $\tau$ for
\eve and \adam, respectively, induces a unique play $\out{\sigma}{\tau}$
consistent with both $\sigma$ and $\tau$.

\paragraph*{Winning strategies.}
In a game with payoff set $P$, the strategy $\sigma$ for \eve is a \emph{winning
strategy} if for all strategies $\tau$ for \adam it holds that
$\out{\sigma}{\tau} \in P$; the strategy $\tau$ for \eve is winning if for all
strategies $\sigma$ for \eve we have $\out{\sigma}{\tau} \not\in P$.

\begin{proposition}[Borel determinacy~\cite{martin75}]
    For all games with Borel-definable payoff set $P$, it holds that there
    exists a winning strategy for \adam in the game if and only if there exists
    no winning strategy for \eve in the game.
\end{proposition}

\subsection{Realizability and synthesis}
The realizability and synthesis problems are defined for games whose payoff sets
are given as the language of an automaton. We sometimes refer to these as
\emph{games played on automata}.
\begin{definition}[Problems]
    Consider finite input and output alphabets $I$ and $O$, respectively, and
    an automaton $\calN$ with alphabet $I \times O$.
    The \emph{realizability problem} asks whether
    there exists a winning strategy for \eve in the game with payoff set 
    $\lang{\calN}$. The \emph{synthesis problem}
    further asks to compute and output such a strategy if one exists.
\end{definition}

\paragraph*{Finite-memory strategies.}
A \emph{finite-memory} strategy $\sigma$ for \eve in a game played on the
automaton $\calN =
(Q,q_0,I \times O,\Delta)$ with finite input and
output alphabets $I$ and $O$ is a strategy that can be encoded as a
\emph{(deterministic) Mealy machine} $\calM = (S,s_0,I, \lambda_u,\lambda_o)$
where $S$ is a finite set of (memory) states, $s_0$ is the initial state,
$\lambda_u : S \times (Q \times I) \to S$ is the update function and $\lambda_o
: S \times (Q \times I) \to O$ is the output function. The machine encodes
$\sigma$ in the following way. For all play prefixes $\langle i_1,  o_1\rangle
\dots \langle i_{k-1}, o_{k-1} \rangle$ and input letters $i_k \in I$ we have
that $\sigma(\langle i_1, o_1\rangle \dots \langle i_{k-1}, o_{k-1} \rangle,
i_k) = \lambda_o(s_k,i_k)$ where $s_{\ell + 1} =
\lambda_u(s_\ell,i_\ell)$. We then say the strategy $\sigma$ has \emph{memory}
$|S|$. In particular, when $|S| = 1$, we say the strategy is \emph{memoryless}
or \emph{positional}.\todo{Do we need similar definitions for \adam?}

\begin{proposition}[Memoryless determinacy for parity games~\cite{ag11}]
    For all games played on deterministic parity automata, it holds that there
    exists a winning strategy for \eve in the game if and only if there exists a
    memoryless winning strategy for \eve.
\end{proposition}

\section{Universal limit-deterministic co-B\"uchi games}
In this section we propose an Acacia-like~\cite{acacia} solution for the
realizability and synthesis problems for games played on universal
limit-deterministic co-B\"uchi games. Recall that these are games whose payoff
set is given by a universal co-B\"uchi (UcB) automaton that is deterministic in
the limit.
From a given UcB automaton we will consider a stronger version of its acceptance
condition which asks that visits to accepting states are bounded.

\begin{definition}[Universal $K$-co-B\"uchi automata]
    A universal $K$-co-B\"uchi (UKcB) automaton is an automaton $\calN =
    (Q,q_0,A,\Delta,B)$ that accepts a word $\alpha$ if and only if all of its
    runs $\rho = q_0 \alpha(1) q_1 \alpha(1) \dots$
    on $\alpha$ visit at most $K$ accepting states, \ie~$|\{i \in \mathbb{N} \st
    q_i \in B\}| \le K$.
\end{definition}

Our intention is to give an incremental algorithm to solve games played on UcB
automata by solving the game played on the same automaton with UKcB semantics
for increasing values of $K$. As a first step, we will construct a finite
``safety automaton'' which keeps track of the maximal accepting-state visit
count per reachable state instead of doing so for each run.
\begin{definition}[Universal safety automaton]
    For a given limit-deterministic UcB automaton $\calN = (Q,q_0,A,\Delta,B)$
    with deterministic state-set $Q_d \subseteq Q$ and a bound $K \in
    \mathbb{N}$, let $\calS^K_\calN$ be the automaton $(Q',q'_0,A,\Delta',F)$ with
    components defined as follows.
    \begin{itemize}
        \item $Q' := \pow(Q) \times \{f : Q_d \to \{0,\dots,K\} \cup
            \{\bot,\top\}\}$,
        \item $q'_0 := \langle \{q_0\}, \{ q \mapsto \bot \st q \in Q_d\}
            \rangle$,\todo{I am assuming it is not deterministic, hence $q_0
            \not\in Q_d$}
        \item $F := \{ \langle s, f \rangle \in Q' \st f(q) = \top \text{ for
            some } q \in Q_d \}$,
        \item $\Delta'$ consists of all triples 
            $(\langle s, f \rangle, a, \langle t, g \rangle)$ such that
            \begin{itemize}
                \item $\forall q \in t, \exists p \in Q : (p,a,q) \in \Delta$ and
                \item $\forall q \in Q_d :
                        g(q) = \max_{(p,a,q) \in \Delta}
                        f(p) + \indicator{F}{q}$,
                    where $\bot < n < \top$, $\bot + n = \bot$, and $\top + n =
                    \top$ for all $n \in \mathbb{N}$.
            \end{itemize}
    \end{itemize}
    The automaton $\calS_\calN$ accepts a word $\alpha$ if and only if it has no
    run $q_0 \alpha(1) q_1 \alpha(2) \dots$ on $\alpha$ which visits a
    state from $F$, \ie~$\forall i \in \mathbb{N} : q_i \not\in F$.
\end{definition}

The following property of universal safety automata is immediate from the
definition.
\begin{lemma}
    For all $\alpha \in A^\omega$ and all $K \in \mathbb{N}$, there exists a
    run $\rho$ on $\alpha$ in $\calN$ which visits more than $K$ B\"uchi states
    if and only if for all $\ell \le K$ there exists a run $\rho'$ on $\alpha$ in
    $\calS^\ell_\calN$ that visits a final state.
\end{lemma}

The above lemma allows us to argue that \eve winning a universal safety game
implies she wins the game played on the limit-deterministic UcB game.
\begin{lemma}
    Consider input and output alphabets $I$ and $O$, a limit-deterministic UcB
    automaton $\calN$ with alphabet $I \times O$, a bound $K \in \mathbb{N}$,
    and the corresponding universal safety automaton $\calS^K_\calN$.  If there
    exists a winning strategy for \eve in the game played on $\calS^K_\calN$
    then there exists a winning strategy for \eve in the game played on
    $\calN$.
\end{lemma}
\begin{proof}
    We will argue the contraposition of the claimed implication holds.
    Note that $\lang{\calN}$ is clearly Borel definable. It follows that the
    game played on $\calN$ is determined and that, by assumption, there exists a
    winning strategy $\tau$ for \adam. Hence, any play $\pi$ consistent with
    $\tau$ will be such that $\pi \not\in \lang{\calN}$ and thus $\pi \not\in
    \lang{\calS^K_\calN}$. So $\tau$ is a winning strategy for \adam in the game
    played on $\calS^K_\calN$.
\qed\end{proof}

Conversely, there exists a big enough bound $K$ such that if \eve wins the game
played on $\calN$ then she also wins the one played on $\calS^K_\calN$.
%
\begin{lemma}
    Consider input and output alphabets $I$ and $O$ and a UcB
    automaton $\calN$ with alphabet $I \times O$ that is deterministic in the
    limit. There exists a bound $K \in
    \mathbb{N}$ exponential with respect to the size of $\calN$ such that for
    the universal safety automaton $\calS^K_\calN$, if there exists a winning
    strategy for \eve in the game played on $\calN$ then there exists a winning
    strategy for \eve in the game played on $\calS^K_\calN$.
\end{lemma}
\begin{proof}[Sketch]
    Let us consider a deterministic parity automaton $\calD$ such that
    $\lang{\calD} = \lang{\calN}$ and let $\sigma$ be the winning strategy for
    \eve in the game played on $\calN$ (and in the game played on $\calD$). By
    memoryless determinacy of parity games, there is a memoryless strategy $\mu$
    for \eve in the game played on $\calD$. Since $\calD$ has exponential size
    with respect to the size of $\calN$, there exists an exponential-memory
    strategy $\mu'$ for \eve in the game played on $\calN$. It follows that if
    any run of $\calN$ on a play $\pi$ consistent with $\mu'$ has visited
    sufficiently many accepting states, the Mealy machine realizing $\mu'$ must
    have revisited some configuration. Hence, by ``pumping'' this behaviour,
    \adam must be able to win against $\mu'$, contradicting the fact that $\mu'$
    is a winning strategy for \eve. We thus have an exponential upper bound on
    visits to accepting states for all runs on plays consistent with $\mu'$. By
    construction of $\calS^K_\calN$ we have that $\mu'$ is a winning strategy for
    \eve in the game played on $\calS^K_\calN$.
\qed\end{proof}

\subsection{Antichain-based algorithm}
TODO: define the order on the states of $\calS^K_\calN$ (subset and
product-ordering on the function image)

\todo[inline]{There are other possible tweaks to the simple safety game solver:
    (algo 2) allow for resets of counters to avoid early alarms (algo 3) have
    the non-deterministic states be actually deterministic --- except for the
    jump to $Q_d$ --- and reset counters upon changing SCC (algo 4) keep
    separate counting functions for each SCC in $Q_d$}


\bibliographystyle{abbrv}
\bibliography{refs}{ }

\end{document}
